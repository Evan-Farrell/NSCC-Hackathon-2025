%===============================================================================
% Title:        coursedata.tex
% Author:       Evan Farrell
% Date:         Jan 17, 2025
% Description:  This file was intended to be edited in order to populate the text fields in main.tex when compiled.
%               Instead of recompiling the LaTeX files, we chose to compile the document once and then edit
%               the text fields of the format using PyPDF2.
%===============================================================================

% Here we pass all the variables in from the python code. I took the pgf code from the below link.
% https://tex.stackexchange.com/questions/37094/what-is-the-recommended-way-to-assign-a-value-to-a- variable-and-retrieve-it-for?
%-------------------------------------------------------- 
\newcommand{\setvalue}[1]{\pgfkeys{/variables/#1}}
\newcommand{\getvalue}[1]{\pgfkeysvalueof{/variables/#1}}
\newcommand{\declare}[1]{%
 \pgfkeys{
  /variables/#1.is family,
  /variables/#1.unknown/.style = {\pgfkeyscurrentpath/\pgfkeyscurrentname/.initial = ##1}
 }
}
\declare{student/}
\declare{term1/}
\declare{term2/}
\declare{term3/}
\declare{term4/}
\declare{term5/}
\declare{term6/}
%-------------------------------------------------------- 
%Student Info
\setvalue{student, name=Evan Farrell,id=W0518150, program = Iot,date=,}
%Term1
\setvalue{term1,
            totalUnitsEnrolled=6,
            totalUnitsFull=6,
            percentFullTimeLoad=100\%,
            code1=DBAS 1000,
            class1=Databases and Stuff, 
            val1=2,
            code2=HACK 1001,
            class2=blah blah blah, 
            val2=7,
            code3=ABCD 2222,
            class3=Alphabets, 
            val3=0,
            code4=,
            class4=class4, 
            val4=,
            code5=,
            class5=class5, 
            val5=,
            code6=,
            class6=class6, 
            val6=,}

%Term2
\setvalue{term2,
            totalUnitsEnrolled=,
            totalUnitsFull=,
            percentFullTimeLoad=,
            code1=DBAS 1000,
            class1=Databases and Stuff, 
            val1=2,
            code2=HACK 1001,
            class2=blah blah blah, 
            val2=7,
            code3=ABCD 2222,
            class3=Alphabets, 
            val3=0,
            code4=,
            class4=, 
            val4=,
            code5=,
            class5=, 
            val5=,
            code6=,
            class6=, 
            val6=,}

%Term3
\setvalue{term3,
            totalUnitsEnrolled=,
            totalUnitsFull=,
            percentFullTimeLoad=,
            code1=DBAS 1000,
            class1=Databases and Stuff, 
            val1=2,
            code2=HACK 1001,
            class2=blah blah blah, 
            val2=7,
            code3=ABCD 2222,
            class3=Alphabets, 
            val3=00,
            code4=,
            class4=, 
            val4=,
            code5=,
            class5=, 
            val5=,
            code6=,
            class6=, 
            val6=,}

%Term4
\setvalue{term4,
            totalUnitsEnrolled=,
            totalUnitsFull=,
            percentFullTimeLoad=,
            code1=HACK1000,
            class1=Databases and Stuff, 
            val1=2,
            code2=term4 2,
            class2=blah blah blah, 
            val2=7,
            code3=term4 3,
            class3=Alphabets, 
            val3=0,
            code4=term4 4,
            class4=, 
            val4=,
            code5=term4 5,
            class5=, 
            val5=,
            code6=term 4 6,
            class6=, 
            val6=,}

%Term5
\setvalue{term5,
            totalUnitsEnrolled=,
            totalUnitsFull=,
            percentFullTimeLoad=,
            code1=DBAS 1000,
            class1=Databases and Stuff, 
            val1=2,
            code2=HACK 1001,
            class2=blah blah blah, 
            val2=7,
            code3=ABCD 2222,
            class3=Alphabets, 
            val3=0,
            code4=,
            class4=, 
            val4=,
            code5=,
            class5=, 
            val5=,
            code6=,
            class6=, 
            val6=,}

%Term6
\setvalue{term6,
            totalUnitsEnrolled=,
            totalUnitsFull=,
            percentFullTimeLoad=,
            code1=DBAS 1000,
            class1=Databases and Stuff, 
            val1=2,
            code2=HACK 1001,
            class2=blah blah blah, 
            val2=7,
            code3=ABCD 2222,
            class3=Alphabets, 
            val3=0,
            code4=,
            class4=, 
            val4=,
            code5=,
            class5=, 
            val5=,
            code6=,
            class6=, 
            val6=,}
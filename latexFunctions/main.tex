\documentclass[legalpaper]{article}
\usepackage{calc,amsmath,amssymb,amsfonts}
\usepackage[T1]{fontenc}
\usepackage[english]{babel}
\usepackage{xcolor,fancyhdr}
\usepackage[top=0.5in,bottom=0.4in,hmargin=0.3in,noheadfoot]{geometry}
\usepackage{array,supertabular,hhline,enumitem,hyperref}
\usepackage{pgfkeys}
%===============================================================================
% Title:        coursedata.tex
% Author:       Evan Farrell
% Date:         Jan 17, 2025
% Description:  This file was intended to be edit to populate the textfields of main.tex when compiled
%               the method we went with was to instead compile the document once then edit the text fields
%               of the format rather than recompile any latex files.
%===============================================================================

% Here we pass all the variables in from the python code. I took the pgf code from the below link.
% https://tex.stackexchange.com/questions/37094/what-is-the-recommended-way-to-assign-a-value-to-a- variable-and-retrieve-it-for?
%-------------------------------------------------------- 
\newcommand{\setvalue}[1]{\pgfkeys{/variables/#1}}
\newcommand{\getvalue}[1]{\pgfkeysvalueof{/variables/#1}}
\newcommand{\declare}[1]{%
 \pgfkeys{
  /variables/#1.is family,
  /variables/#1.unknown/.style = {\pgfkeyscurrentpath/\pgfkeyscurrentname/.initial = ##1}
 }
}
\declare{student/}
\declare{term1/}
\declare{term2/}
\declare{term3/}
\declare{term4/}
\declare{term5/}
\declare{term6/}
%-------------------------------------------------------- 
%Student Info
\setvalue{student, name=Evan Farrell,id=W0518150, program = Iot,date=,}
%Term1
\setvalue{term1,
            totalUnitsEnrolled=6,
            totalUnitsFull=6,
            percentFullTimeLoad=100\%,
            code1=DBAS 1000,
            class1=Databases and Stuff, 
            val1=2,
            code2=HACK 1001,
            class2=blah blah blah, 
            val2=7,
            code3=ABCD 2222,
            class3=Alphabets, 
            val3=0,
            code4=,
            class4=class4, 
            val4=,
            code5=,
            class5=class5, 
            val5=,
            code6=,
            class6=class6, 
            val6=,}

%Term2
\setvalue{term2,
            totalUnitsEnrolled=,
            totalUnitsFull=,
            percentFullTimeLoad=,
            code1=DBAS 1000,
            class1=Databases and Stuff, 
            val1=2,
            code2=HACK 1001,
            class2=blah blah blah, 
            val2=7,
            code3=ABCD 2222,
            class3=Alphabets, 
            val3=0,
            code4=,
            class4=, 
            val4=,
            code5=,
            class5=, 
            val5=,
            code6=,
            class6=, 
            val6=,}

%Term3
\setvalue{term3,
            totalUnitsEnrolled=,
            totalUnitsFull=,
            percentFullTimeLoad=,
            code1=DBAS 1000,
            class1=Databases and Stuff, 
            val1=2,
            code2=HACK 1001,
            class2=blah blah blah, 
            val2=7,
            code3=ABCD 2222,
            class3=Alphabets, 
            val3=00,
            code4=,
            class4=, 
            val4=,
            code5=,
            class5=, 
            val5=,
            code6=,
            class6=, 
            val6=,}

%Term4
\setvalue{term4,
            totalUnitsEnrolled=,
            totalUnitsFull=,
            percentFullTimeLoad=,
            code1=HACK1000,
            class1=Databases and Stuff, 
            val1=2,
            code2=term4 2,
            class2=blah blah blah, 
            val2=7,
            code3=term4 3,
            class3=Alphabets, 
            val3=0,
            code4=term4 4,
            class4=, 
            val4=,
            code5=term4 5,
            class5=, 
            val5=,
            code6=term 4 6,
            class6=, 
            val6=,}

%Term5
\setvalue{term5,
            totalUnitsEnrolled=,
            totalUnitsFull=,
            percentFullTimeLoad=,
            code1=DBAS 1000,
            class1=Databases and Stuff, 
            val1=2,
            code2=HACK 1001,
            class2=blah blah blah, 
            val2=7,
            code3=ABCD 2222,
            class3=Alphabets, 
            val3=0,
            code4=,
            class4=, 
            val4=,
            code5=,
            class5=, 
            val5=,
            code6=,
            class6=, 
            val6=,}

%Term6
\setvalue{term6,
            totalUnitsEnrolled=,
            totalUnitsFull=,
            percentFullTimeLoad=,
            code1=DBAS 1000,
            class1=Databases and Stuff, 
            val1=2,
            code2=HACK 1001,
            class2=blah blah blah, 
            val2=7,
            code3=ABCD 2222,
            class3=Alphabets, 
            val3=0,
            code4=,
            class4=, 
            val4=,
            code5=,
            class5=, 
            val5=,
            code6=,
            class6=, 
            val6=,}

\hypersetup{colorlinks=true,allcolors=blue,pdfauthor=Farrell;Evan}
% Outline numbering
\setcounter{secnumdepth}{0}
\makeatletter
\newcommand\arraybslash{\let\\\@arraycr}
\makeatother

\fancypagestyle{Standard}{\fancyhf{}
  \fancyhead[L]{}
  \fancyfoot[L]{}
  \renewcommand\headrulewidth{0pt}
  \renewcommand\footrulewidth{0pt}
  \renewcommand\thepage{\arabic{page}}
}
\pagestyle{Standard}
\setlength\tabcolsep{1mm}
\renewcommand\arraystretch{1.5}




            
% \setvalue{term1, property 1 = value 1, property 2 = value 2}
% \setvalue{term1, property 1 = value 1, property 2 = value 2}
% \setvalue{term1, property 1 = value 1, property 2 = value 2}
% \setvalue{term1, property 1 = value 1, property 2 = value 2}
% \setvalue{term1, property 1 = value 1, property 2 = value 2}

% \setvalue{term = foo foo bar}
% \getvalue{VARIABLE1}
% \declare{}

\begin{document}

% HEADER FOR TABLES
%-------------------------------------------------------- 
\pagestyle{Standard}
{\centering\textbf{Alternate Pathway Plan}\par}
\begin{flushleft}
\textbf{Student Name:} \TextField[name=name, width=3in, charsize=12pt, value=\getvalue{student/name}]{} 
\hspace{1in}
\textbf{ID:}  \TextField[name=id, width=1.5in, charsize=12pt, value=\getvalue{student/id}]{} 
\\ \vspace{5mm} 
\textbf{Program:} \TextField[name=prog, width=4in, charsize=12pt, value=\getvalue{student/program}]{} 
\hspace{0.38in}
\textbf{Date:}  \TextField[name=date, width=1.4in, charsize=12pt, value=\getvalue{student/date}]{} 
\end{flushleft}
%-------------------------------------------------------- 

\begin{flushleft}

\begin{supertabular}{|p{1in}|p{2.2in}|m{0.4in}||p{1in}|p{2.2in}|m{0.4in}|}
\hline
\multicolumn{3}{|m{3.6in}||}{\centering \textbf{Session:\TextField[name=session1, width=1.4in, charsize=12pt, value=]}} &
\multicolumn{3}{m{3.6in}|}{\centering \textbf{Session:\TextField[name=session2, width=1.4in, charsize=12pt, value=]}}\\\hline
\centering \textbf{Course} &
\centering\textbf{Description}  &
\centering \textbf{Unit value} &
\centering \textbf{Course} &
\centering \textbf{Description} &
\centering\arraybslash \textbf{Unit value}\\\hline
~\TextField[name=code11, width=1in, charsize=12pt,value=\getvalue{term1/code1}]{}
 &
~ \TextField[name=class11, width=2in, charsize=10pt, value=\getvalue{term1/class1}]{}
 &
~ \TextField[name=val11, width=0.3in, charsize=12pt, value=\getvalue{term1/val1}]{}
 &
~\TextField[name=code21, width=1in, charsize=12pt,value=\getvalue{term2/code1}]{}
 &
~ \TextField[name=class21, width=2in, charsize=10pt, value=\getvalue{term2/class1}]{}
 &
~ \TextField[name=val21, width=0.3in, charsize=12pt, value=\getvalue{term2/val1}]{}
\\\hline
~\TextField[name=code12, width=1in, charsize=12pt,value=\getvalue{term1/code2}]{}
 &
~ \TextField[name=class12, width=2in, charsize=10pt, value=\getvalue{term1/class2}]{}
 &
~ \TextField[name=val12, width=0.3in, charsize=12pt, value=\getvalue{term1/val2}]{}
 &
~\TextField[name=code22, width=1in, charsize=12pt,value=\getvalue{term2/code2}]{}
 &
~ \TextField[name=class22, width=2in, charsize=10pt, value=\getvalue{term2/class2}]{}
 &
~ \TextField[name=val22, width=0.3in, charsize=12pt, value=\getvalue{term2/val2}]{}
\\\hline
~\TextField[name=code13, width=1in, charsize=12pt,value=\getvalue{term1/code3}]{}
 &
~ \TextField[name=class13, width=2in, charsize=10pt, value=\getvalue{term1/class3}]{}
 &
~ \TextField[name=val13, width=0.3in, charsize=12pt, value=\getvalue{term1/val3}]{}
 &
~\TextField[name=code23, width=1in, charsize=12pt,value=\getvalue{term2/code3}]{}
 &
~ \TextField[name=class23, width=2in, charsize=10pt, value=\getvalue{term2/class3}]{}
 &
~ \TextField[name=val23, width=0.3in, charsize=12pt, value=\getvalue{term2/val3}]{}
\\\hline
~\TextField[name=code14, width=1in, charsize=12pt,value=\getvalue{term1/code4}]{}
 &
~ \TextField[name=class14, width=2in, charsize=10pt, value=\getvalue{term1/class4}]{}
 &
~ \TextField[name=val14, width=0.3in, charsize=12pt, value=\getvalue{term1/val4}]{}
 &
~\TextField[name=code24, width=1in, charsize=12pt,value=\getvalue{term2/code4}]{}
 &
~ \TextField[name=class24, width=2in, charsize=10pt, value=\getvalue{term2/class4}]{}
 &
~ \TextField[name=val24, width=0.3in, charsize=12pt, value=\getvalue{term2/val4}]{}
\\\hline
~\TextField[name=code15, width=1in, charsize=12pt,value=\getvalue{term1/code5}]{}
 &
~ \TextField[name=class15, width=2in, charsize=10pt, value=\getvalue{term1/class5}]{}
 &
~ \TextField[name=val15, width=0.3in, charsize=12pt, value=\getvalue{term1/val5}]{}
 &
~\TextField[name=code25, width=1in, charsize=12pt,value=\getvalue{term2/code5}]{}
 &
~ \TextField[name=class25, width=2in, charsize=10pt, value=\getvalue{term2/class5}]{}
 &
~ \TextField[name=val25, width=0.3in, charsize=12pt, value=\getvalue{term2/val5}]{}
\\\hline
~\TextField[name=code16, width=1in, charsize=12pt,value=\getvalue{term1/code6}]{}
 &
~ \TextField[name=class16, width=2in, charsize=10pt, value=\getvalue{term1/class6}]{}
 &
~ \TextField[name=val16, width=0.3in, charsize=12pt, value=\getvalue{term1/val6}]{}
 &
~\TextField[name=code26, width=1in, charsize=12pt,value=\getvalue{term2/code6}]{}
 &
~ \TextField[name=class26, width=2in, charsize=10pt, value=\getvalue{term2/class6}]{}
 &
~ \TextField[name=val26, width=0.3in, charsize=12pt, value=\getvalue{term2/val6}]{}
\\\hline
\multicolumn{2}{|p{3in}|}{Total units for courses enrolled} &
~\TextField[name=total1, width=0.3in, charsize=12pt, value=\getvalue{term1/totalUnitsEnrolled}]{}
 &
\multicolumn{2}{p{3in}|}{Total units for courses enrolled} &
~\TextField[name=total2, width=0.3in, charsize=12pt, value=\getvalue{term2/totalUnitsEnrolled}]{}
\\\hline
\multicolumn{2}{|p{3in}|}{Total units for full semester} &
~\TextField[name=full1, width=0.3in, charsize=12pt, value=\getvalue{term1/totalUnitsFull}]{}
 &
\multicolumn{2}{p{3in}|}{Total units for full semester} &
~\TextField[name=full2, width=0.3in, charsize=12pt, value=\getvalue{term2/totalUnitsFull}]{}
\\\hline
\multicolumn{2}{|p{3in}|}{\% of FT course load} &
~\TextField[name=percent1, width=0.3in, charsize=12pt, value=\getvalue{term1/percentFullTimeLoad}]{}
 &
\multicolumn{2}{p{3in}|}{\% of FT course load} &
~\TextField[name=percent2, width=0.3in, charsize=12pt, value=\getvalue{term2/percentFullTimeLoad}]{}
\\\hline
\end{supertabular}
\end{flushleft}

\begin{flushleft}

\begin{supertabular}{|p{1in}|p{2.2in}|m{0.4in}||p{1in}|p{2.2in}|m{0.4in}|}
\hline
\multicolumn{3}{|m{3.6in}||}{\centering \textbf{Session:\TextField[name=session3, width=1.4in, charsize=12pt, value=]}} &
\multicolumn{3}{m{3.6in}|}{\centering \textbf{Session:\TextField[name=session4, width=1.4in, charsize=12pt, value=]}}\\\hline
\centering \textbf{Course} &
\centering\textbf{Description}  &
\centering \textbf{Unit value} &
\centering \textbf{Course} &
\centering \textbf{Description} &
\centering\arraybslash \textbf{Unit value}\\\hline
~\TextField[name=code31, width=1in, charsize=12pt,value=\getvalue{term3/code1}]{}
 &
~ \TextField[name=class31, width=2in, charsize=10pt, value=\getvalue{term3/class1}]{}
 &
~ \TextField[name=val31, width=0.3in, charsize=12pt, value=\getvalue{term3/val1}]{}
 &
~\TextField[name=code41, width=1in, charsize=12pt,value=\getvalue{term4/code1}]{}
 &
~ \TextField[name=class41, width=2in, charsize=10pt, value=\getvalue{term4/class1}]{}
 &
~ \TextField[name=val41, width=0.3in, charsize=12pt, value=\getvalue{term4/val1}]{}
\\\hline
~\TextField[name=code32, width=1in, charsize=12pt,value=\getvalue{term3/code2}]{}
 &
~ \TextField[name=class32, width=2in, charsize=10pt, value=\getvalue{term3/class2}]{}
 &
~ \TextField[name=val32, width=0.3in, charsize=12pt, value=\getvalue{term3/val2}]{}
 &
~\TextField[name=code42, width=1in, charsize=12pt,value=\getvalue{term4/code2}]{}
 &
~ \TextField[name=class42, width=2in, charsize=10pt, value=\getvalue{term4/class2}]{}
 &
~ \TextField[name=val42, width=0.3in, charsize=12pt, value=\getvalue{term4/val2}]{}
\\\hline
~\TextField[name=code33, width=1in, charsize=12pt,value=\getvalue{term3/code3}]{}
 &
~ \TextField[name=class33, width=2in, charsize=10pt, value=\getvalue{term3/class3}]{}
 &
~ \TextField[name=val33, width=0.3in, charsize=12pt, value=\getvalue{term3/val3}]{}
 &
~\TextField[name=code43, width=1in, charsize=12pt,value=\getvalue{term4/code3}]{}
 &
~ \TextField[name=class43, width=2in, charsize=10pt, value=\getvalue{term4/class3}]{}
 &
~ \TextField[name=val43, width=0.3in, charsize=12pt, value=\getvalue{term4/val3}]{}
\\\hline
~\TextField[name=code34, width=1in, charsize=12pt,value=\getvalue{term3/code4}]{}
 &
~ \TextField[name=class34, width=2in, charsize=10pt, value=\getvalue{term3/class4}]{}
 &
~ \TextField[name=val34, width=0.3in, charsize=12pt, value=\getvalue{term3/val4}]{}
 &
~\TextField[name=code44, width=1in, charsize=12pt,value=\getvalue{term4/code4}]{}
 &
~ \TextField[name=class44, width=2in, charsize=10pt, value=\getvalue{term4/class4}]{}
 &
~ \TextField[name=val44, width=0.3in, charsize=12pt, value=\getvalue{term4/val4}]{}
\\\hline
~\TextField[name=code35, width=1in, charsize=12pt,value=\getvalue{term3/code5}]{}
 &
~ \TextField[name=class35, width=2in, charsize=10pt, value=\getvalue{term3/class5}]{}
 &
~ \TextField[name=val35, width=0.3in, charsize=12pt, value=\getvalue{term3/val5}]{}
 &
~\TextField[name=code45, width=1in, charsize=12pt,value=\getvalue{term4/code5}]{}
 &
~ \TextField[name=class45, width=2in, charsize=10pt, value=\getvalue{term4/class5}]{}
 &
~ \TextField[name=val45, width=0.3in, charsize=12pt, value=\getvalue{term4/val5}]{}
\\\hline
~\TextField[name=code36, width=1in, charsize=12pt,value=\getvalue{term3/code6}]{}
 &
~ \TextField[name=class36, width=2in, charsize=10pt, value=\getvalue{term3/class6}]{}
 &
~ \TextField[name=val36, width=0.3in, charsize=12pt, value=\getvalue{term3/val6}]{}
 &
~\TextField[name=code46, width=1in, charsize=12pt,value=\getvalue{term4/code6}]{}
 &
~ \TextField[name=class46, width=2in, charsize=10pt, value=\getvalue{term4/class6}]{}
 &
~ \TextField[name=val46, width=0.3in, charsize=12pt, value=\getvalue{term4/val6}]{}
\\\hline
\multicolumn{2}{|p{3in}|}{Total units for courses enrolled} &
~\TextField[name=total3, width=0.3in, charsize=12pt, value=\getvalue{term1/totalUnitsEnrolled}]{}
 &
\multicolumn{2}{p{3in}|}{Total units for courses enrolled} &
~\TextField[name=total4, width=0.3in, charsize=12pt, value=\getvalue{term2/totalUnitsEnrolled}]{}
\\\hline
\multicolumn{2}{|p{3in}|}{Total units for full semester} &
~\TextField[name=full3, width=0.3in, charsize=12pt, value=\getvalue{term1/totalUnitsFull}]{}
 &
\multicolumn{2}{p{3in}|}{Total units for full semester} &
~\TextField[name=full4, width=0.3in, charsize=12pt, value=\getvalue{term2/totalUnitsFull}]{}
\\\hline
\multicolumn{2}{|p{3in}|}{\% of FT course load} &
~\TextField[name=percent3, width=0.3in, charsize=12pt, value=\getvalue{term1/percentFullTimeLoad}]{}
 &
\multicolumn{2}{p{3in}|}{\% of FT course load} &
~\TextField[name=percent4, width=0.3in, charsize=12pt, value=\getvalue{term2/percentFullTimeLoad}]{}
\\\hline
\end{supertabular}
\end{flushleft}

\begin{flushleft}

\begin{supertabular}{|p{1in}|p{2.2in}|m{0.4in}||p{1in}|p{2.2in}|m{0.4in}|}
\hline
\multicolumn{3}{|m{3.6in}||}{\centering \textbf{Session:\TextField[name=session5, width=1.4in, charsize=12pt, value=]}} &
\multicolumn{3}{m{3.6in}|}{\centering \textbf{Session:\TextField[name=session6, width=1.4in, charsize=12pt, value=]}}\\\hline
\centering \textbf{Course} &
\centering\textbf{Description}  &
\centering \textbf{Unit value} &
\centering \textbf{Course} &
\centering \textbf{Description} &
\centering\arraybslash \textbf{Unit value}\\\hline
~\TextField[name=code51, width=1in, charsize=12pt,value=\getvalue{term5/code1}]{}
 &
~ \TextField[name=class51, width=2in, charsize=10pt, value=\getvalue{term5/class1}]{}
 &
~ \TextField[name=val51, width=0.3in, charsize=12pt, value=\getvalue{term5/val1}]{}
 &
~\TextField[name=code61, width=1in, charsize=12pt,value=\getvalue{term6/code1}]{}
 &
~ \TextField[name=class61, width=2in, charsize=10pt, value=\getvalue{term6/class1}]{}
 &
~ \TextField[name=val61, width=0.3in, charsize=12pt, value=\getvalue{term6/val1}]{}
\\\hline
~\TextField[name=code52, width=1in, charsize=12pt,value=\getvalue{term5/code2}]{}
 &
~ \TextField[name=class52, width=2in, charsize=10pt, value=\getvalue{term5/class2}]{}
 &
~ \TextField[name=val52, width=0.3in, charsize=12pt, value=\getvalue{term5/val2}]{}
 &
~\TextField[name=code62, width=1in, charsize=12pt,value=\getvalue{term6/code2}]{}
 &
~ \TextField[name=class62, width=2in, charsize=10pt, value=\getvalue{term6/class2}]{}
 &
~ \TextField[name=val62, width=0.3in, charsize=12pt, value=\getvalue{term6/val2}]{}
\\\hline
~\TextField[name=code53, width=1in, charsize=12pt,value=\getvalue{term5/code3}]{}
 &
~ \TextField[name=class53, width=2in, charsize=10pt, value=\getvalue{term5/class3}]{}
 &
~ \TextField[name=val53, width=0.3in, charsize=12pt, value=\getvalue{term5/val3}]{}
 &
~\TextField[name=code63, width=1in, charsize=12pt,value=\getvalue{term6/code3}]{}
 &
~ \TextField[name=class63, width=2in, charsize=10pt, value=\getvalue{term6/class3}]{}
 &
~ \TextField[name=val63, width=0.3in, charsize=12pt, value=\getvalue{term6/val3}]{}
\\\hline
~\TextField[name=code54, width=1in, charsize=12pt,value=\getvalue{term5/code4}]{}
 &
~ \TextField[name=class54, width=2in, charsize=10pt, value=\getvalue{term5/class4}]{}
 &
~ \TextField[name=val54, width=0.3in, charsize=12pt, value=\getvalue{term5/val4}]{}
 &
~\TextField[name=code64, width=1in, charsize=12pt,value=\getvalue{term6/code4}]{}
 &
~ \TextField[name=class64, width=2in, charsize=10pt, value=\getvalue{term6/class4}]{}
 &
~ \TextField[name=val64, width=0.3in, charsize=12pt, value=\getvalue{term6/val4}]{}
\\\hline
~\TextField[name=code55, width=1in, charsize=12pt,value=\getvalue{term5/code5}]{}
 &
~ \TextField[name=class55, width=2in, charsize=10pt, value=\getvalue{term5/class5}]{}
 &
~ \TextField[name=val55, width=0.3in, charsize=12pt, value=\getvalue{term5/val5}]{}
 &
~\TextField[name=code65, width=1in, charsize=12pt,value=\getvalue{term6/code5}]{}
 &
~ \TextField[name=class65, width=2in, charsize=10pt, value=\getvalue{term6/class5}]{}
 &
~ \TextField[name=val65, width=0.3in, charsize=12pt, value=\getvalue{term6/val5}]{}
\\\hline
~\TextField[name=code56, width=1in, charsize=12pt,value=\getvalue{term5/code6}]{}
 &
~ \TextField[name=class56, width=2in, charsize=10pt, value=\getvalue{term5/class6}]{}
 &
~ \TextField[name=val56, width=0.3in, charsize=12pt, value=\getvalue{term5/val6}]{}
 &
~\TextField[name=code66, width=1in, charsize=12pt,value=\getvalue{term6/code6}]{}
 &
~ \TextField[name=class66, width=2in, charsize=10pt, value=\getvalue{term6/class6}]{}
 &
~ \TextField[name=val66, width=0.3in, charsize=12pt, value=\getvalue{term6/val6}]{}
\\\hline
\multicolumn{2}{|p{3in}|}{Total units for courses enrolled} &
~\TextField[name=total5, width=0.3in, charsize=12pt, value=\getvalue{term5/totalUnitsEnrolled}]{}
 &
\multicolumn{2}{p{3in}|}{Total units for courses enrolled} &
~\TextField[name=total6, width=0.3in, charsize=12pt, value=\getvalue{term6/totalUnitsEnrolled}]{}
\\\hline
\multicolumn{2}{|p{3in}|}{Total units for full semester} &
~\TextField[name=full5, width=0.3in, charsize=12pt, value=\getvalue{term5/totalUnitsFull}]{}
 &
\multicolumn{2}{p{3in}|}{Total units for full semester} &
~\TextField[name=full6, width=0.3in, charsize=12pt, value=\getvalue{term6/totalUnitsFull}]{}
\\\hline
\multicolumn{2}{|p{3in}|}{\% of FT course load} &
~\TextField[name=percent5, width=0.3in, charsize=12pt, value=\getvalue{term5/percentFullTimeLoad}]{}
 &
\multicolumn{2}{p{3in}|}{\% of FT course load} &
~\TextField[name=percent6, width=0.3in, charsize=12pt, value=\getvalue{term6/percentFullTimeLoad}]{}
\\\hline
\end{supertabular}
\end{flushleft}

\bigskip

\end{document}
